\section{Listings}
\def\Listings{\texorpdfstring{\T{\link{https://www.ctan.org/pkg/listings}{listings}}}{\T{listings}}\xspace}
\def\Minted{\texorpdfstring{\T{\link{https://www.ctan.org/pkg/minted}{minted}}}{\T{minted}}\xspace}
\ltxpreview{MyFirstListing}
\begin{tcboutputlisting}
\documentclass{article}
\usepackage{listings}

\begin{document}
\begin{lstlisting}[language=java]
public static void main(String[] args) {
   System.out.println("Hello World!");
}
\end{lstlisting}
\end{document}
\end{tcboutputlisting}

\subsection[The listings Package]{The \Listings Package}
\begin{frame}{Typesetting code with the \Listings package}
   \begin{itemize}
      \item As some characters have a special meaning in \LaTeX{} (\T{\textbackslash}, \T{\#}, spaces~\ldots), we cannot just \say{paste} source code
      \item With the \Listings package however, we can typeset source code in a very flexible way!
      \item Let's look at an example with the \blatex{lstlisting} environment:
      \eblLoadLtx[deletekeywords={[3]{java}}]{MyFirstListing}{graphics options={trim=4cm 23.25cm 6cm 4cm, clip}}{righthand width=.4\linewidth}
   \end{itemize}
\end{frame}

\ltxpreview{MyTTListing}
\begin{tcboutputlisting}
\documentclass{article}
\usepackage{xcolor,listings}
\lstset{
   basicstyle=\ttfamily,
   stringstyle=\color{teal}
}

\begin{document}
\begin{lstlisting}[language=java]
public static void main(String[] args) {
   System.out.println("Hello World!");
}
\end{lstlisting}
\end{document}
\end{tcboutputlisting}

\begin{frame}{Changing the style of the code}
   \begin{itemize}
      \item To get the common teletype-font, we can change the \T{basicstyle} to use the \blatex{\\ttfamily}
      \item As several options will remain the same for all (or at least most) listings, we can set them with \blatex{\\lstset\{<options>\}}
      \item Similarly, we can make use of colors
      \lstfs{9}\eblLoadLtx[deletekeywords={[3]{java}}]{MyTTListing}{graphics options={trim=4cm 21.75cm 6cm 4cm, clip}}{righthand width=.4\linewidth}
   \end{itemize}
\end{frame}

\ltxpreview{MyFontListing}
\begin{tcboutputlisting}
\documentclass{article}
\usepackage[T1]{fontenc}
\usepackage{xcolor,listings}
\usepackage[scaled]{beramono}
\lstset{basicstyle=\ttfamily,
   stringstyle=\color{teal}}

\begin{document}
\begin{lstlisting}[language=java]
public static void main(String[] args) {
   System.out.println("Hello World!");
}
\end{lstlisting}
\end{document}
\end{tcboutputlisting}

\savebox\TempBoxA{\fancyqr[l color=kit-green!60!black,r color=kit-green!60!kit-blue!60!black,height=3.75cm,level=H]{https://tug.org/FontCatalogue/}}
\begin{frame}{Changing the font}
   \begin{itemize}
      \item Sadly, the default teletypefont does not come with a boldface variant \begin{itemize}
         \item We can change it by simply loading a package (e.g., \T{\link{https://www.ctan.org/pkg/bera}{beramono}})
         \item Be sure to set the font-encoding correctly (e.g., \T{T1})
      \end{itemize}
      \item For a huge list of font-packages see the \link{https://tug.org/FontCatalogue/}{{\LaTeX}, font catalogue}
      \lstfs{9}\eblLoadLtx[deletekeywords={[3]{java}}]{MyFontListing}{graphics options={trim=4cm 21.75cm 6cm 4cm, clip}}{righthand width=.4\linewidth,before lower={\vspace*{-3\baselineskip}}}
   \end{itemize}
   \begin{tikzpicture}[@O]
      \node[above left=2mm,yshift=1.25cm] at(current page.south east) {\scalebox{.4}{\usebox\TempBoxA}};
   \end{tikzpicture}
\end{frame}

\begin{frame}{Where to look for more}
   \begin{itemize}
      \itemsep8pt
      \item The \link{http://mirrors.ctan.org/macros/latex/contrib/listings/listings.pdf}{documentation} gives a great overview of the packages' capabilities \begin{itemize}
         \item You can add your own language with \solGet{command}{\textbackslash lstdefinelanguage}
         \item You can create your own code-environment (e.g., with defaults) using \solGet{command}{\textbackslash lstnewenvironment}
         \item You can create a letter shortcut with \solGet{command}{\textbackslash lstMakeShortInline}
         \item With \T{numbers=left} you can add line numbers on the left side
         \item \T{showstringspaces=false} removes the \say{\T{\textvisiblespace}} symbols within strings
         \item All code shown in this presentation is typeset with this package
      \end{itemize}
      \item If you want to typeset umlauts or perform replacements, have a look at literates and \link{https://en.wikibooks.org/wiki/LaTeX/Source_Code_Listings\#Encoding_issue}{wikibooks}
   \end{itemize}
\end{frame}


\subsection[The minted Package]{The \Minted Package}

\ltxpreview{MyMintedListing}
\begin{tcboutputlisting}
\documentclass{article}
\usepackage{minted}

\begin{document}
\begin{minted}{java}
public static void main(String[] args) {
   System.out.println("Hello World!");
}
\end{minted}
\end{document}
\end{tcboutputlisting}

\begin{frame}{The \Minted package}
   \begin{itemize}
      \item The \Minted uses \T{\link{https://pygments.org/}{pygments}} to highlight source code
      \item As this package requires to call an external program (which has to be installed for that), you need to enable \textit{-\kern0pt-shell-escape} (if you use an IDE you may have to search for the option)
      \item Besides that, the basic workflow is rather similar
      \eblLoadLtx[deletekeywords={[3]{java}}]{MyMintedListing}{graphics options={trim=4cm 21.75cm 6cm 4cm, clip}}{righthand width=.4\linewidth}
   \end{itemize}
\end{frame}

\begin{frame}{\Minted vs. \Listings}
   \begin{itemize}
      \itemsep8pt
      \item While \Minted has more powerful presets (by using \T{\link{https://pygments.org/}{pygments}}), it is way harder to customize
      \item Especially if you want to insert arbitrary markers or perform replacements, \Listings allows this far simpler
      \item Yet, play with both and use the one that fits your needs best\\
         (the author of this introduction prefers \Listings)
   \end{itemize}
\end{frame}