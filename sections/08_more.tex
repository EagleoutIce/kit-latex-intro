\section{More}

\begin{frame}[plain,c]{}
   \vspace*{4.5em}\centering\Huge\bfseries There is so much more!
\end{frame}


\subsection{Bibliographies}

\ltxpreview{MyLittleBib}
\begin{tcboutputlisting}
\documentclass{article}
\usepackage[backend=biber]{biblatex}

\addbibresource{data/example.bib}

\begin{document}
Hello World~\cite{Breckel2021}.
\printbibliography
\end{document}
\end{tcboutputlisting}

\begin{frame}{Bibliographies}
   \begin{itemize}
      \item Citing is important, especially for scientific articles
      \item For this, have a look at the \T{\link{https://www.ctan.org/pkg/biblatex}{biblatex}} package (you have to run \solGet{keywordA}{biber})
      \eblLoadLtx{MyLittleBib}{graphics options={trim=4cm 21.5cm 3cm 4cm, clip}}{righthand width=.5\linewidth,run biber, run pdflatex={--shell-escape}}
      \item There are tools to create and manage these bibliographies (\link{https://www.zotero.org/}{Zotero}, \link{https://www.jabref.org/}{JabRef},~\ldots)
      \item These reference work with \HyperrefPkg (and there is so much more)
   \end{itemize}
\end{frame}


\subsection{Making Presentations}

\ltxpreview{MyLittleBeamer}
\begin{tcboutputlisting}
\documentclass[aspectratio=169]{beamer}
\usetheme{CambridgeUS}
\title{Amazing Presentation}
\author{Florian Sihler}
\date{Yesterday}

\begin{document}
\maketitle
\begin{frame}{Hello World}
   \begin{itemize}
      \item Hello
      \item World
   \end{itemize}
\end{frame}
\end{document}
\end{tcboutputlisting}

\begin{frame}{Presentations --- with the \BeamerCls document class}
   \begin{itemize}
      \item The document you are seeing is written in \LaTeX{} using the \BeamerCls-based template from the \link{https://sdq.kastel.kit.edu/wiki/Dokumentvorlagen}{SDQ}.
      \item \BeamerCls is really powerful and allows for animations, multiple document versions, and more\medskip
      \lstfs{9}\eblLoadLtx{MyLittleBeamer}{}{righthand width=.3\linewidth}
   \end{itemize}
\end{frame}

\subsection{Ti\textit{k}Z --- The Behemoth}

\ltxpreview{MyFirstTikz}
\begin{tcboutputlisting}
\documentclass{article}
\usepackage{tikz}

\begin{document}
\begin{tikzpicture}
   \draw (0,0) circle[radius=2cm];
   \draw[cyan,->] (0,0) -- (2,0)
      node[right] {this is the point!};
   \node[below] at (current bounding box.south) {Hey!};
\end{tikzpicture}
\end{document}
\end{tcboutputlisting}

\begin{frame}{Ti\textit{k}Z --- the behemoth}
   \begin{itemize}
      \item How to draw anything? \TikzPkg is the way to go \begin{itemize}
         \item This monster of a package allows a lot of drawing in \LaTeX{} it has an incredible learning curve
         \item Want to plot something? use \PgfplotsPkg
         \item Want boxes? use \TcolorboxPkg
      \end{itemize}
      \lstfs{9}\eblLoadLtx{MyFirstTikz}{graphics options={trim=4.25cm 19cm 9.5cm 4cm, clip}}{righthand width=.35\linewidth}
   \end{itemize}
\end{frame}

\begin{frame}{Even the penguins!}
   \begin{itemize}
      \itemsep8pt
      \item All the little penguins (\raisebox{-3pt}{\resizebox!\baselineskip{\tikz{\pingu[feet=sit,eyes shiny,conical hat]}}}) you see are made with \TikzPkg
      \item Interested? See the \TikzpingusPkg
      \item Actively developed on \link{https://github.com/EagleoutIce/tikzpingus}{GitHub} (by me, the author)
   \end{itemize}
   \begin{center}
      \tikz{\pingu[eyes shiny,eye patch right, glasses, right wing grab, cup,heart, body type=legacy,left wing wave, horse left]}
   \end{center}
\end{frame}

\subsection{Changing the Document}
\ltxpreview{MyAmazingGeometry}
\begin{tcboutputlisting}
\documentclass{article}
\usepackage[margin=5mm,paper=a5paper,
      landscape,showframe]{geometry}

\begin{document}
Hello World!
\end{document}
\end{tcboutputlisting}
\begin{frame}{The \GeometryPkg package}
   \begin{itemize}
      \item If you want to change the margins, the paper size, or the page layout, you can use the \GeometryPkg package\medskip

      \eblLoadLtx{MyAmazingGeometry}{}{righthand width=.4\linewidth}
   \end{itemize}
\end{frame}

\ltxpreview{MyFunWithParindent}
\begin{tcboutputlisting}
\documentclass{article}
\usepackage{parskip}

\begin{document}
This is a paragraph!

And this is another one!
No indentation, just vertical space!
\end{document}
\end{tcboutputlisting}
\begin{frame}{The \ParskipPkg package}
   \begin{itemize}
      \item Do you dislike the indentation performed at each paragraph (\blatex{\\parindent})?
      \item Consider the \ParskipPkg package which exchanges the horizontal indentation for a vertical one
      \eblLoadLtx{MyFunWithParindent}{graphics options={trim=4.25cm 24cm 6cm 4cm, clip}}{righthand width=.45\linewidth}
      % \item Or you can set \blatex{\\setlength\{\\parindent\}\{0pt\}} and \blatex{\\setlength\{\\parskip\}\{\baselineskip\}} manually.
   \end{itemize}
\end{frame}

\subsection{Custom Macros and More}
\ltxpreview{MyOwnMacro}
\begin{tcboutputlisting}
\documentclass{article}
\usepackage{xcolor}

\newcommand{\highlight}[1]{\textbf{#1ed text}}

\begin{document}
This is a \highlight{highlight}!
\end{document}
\end{tcboutputlisting}

\begin{frame}{Defining your own macros}
\begin{itemize}
   \itemsep4pt
   \item For now, we have only used what \LaTeX{} and other packages provided us
   \item \blatex{\\newcommand} however allows you to define your own commands!
   \item These commands can take numbered arguments and work similar to macro-replacements \begin{itemize}
      \item \blatex{\\newcommand[<number-of-arguments>]\{<body>\}}
      \item You can refer to the arguments as \blatex{\#1}, \blatex{\#2}, \blatex{\#3},~\ldots
   \end{itemize}
   \item As basically everything is macro in \LaTeX{} the superficiality of this overview hurts the author mentally~--- \LaTeX{} is turing complete!
   \lstfs{9}\eblLoadLtx{MyOwnMacro}{graphics options={trim=4.25cm 23cm 11cm 4cm, clip}}{righthand width=.35\linewidth}
\end{itemize}
\end{frame}

\begin{frame}{And macros?}
   \begin{center}
      Please inform yourself about how you can use macros to your advantage.\smallskip

      The fun of \LaTeX{} starts with defining your own!
   \end{center}
\end{frame}

\begin{frame}{Other \LaTeX{} formats}
   \begin{itemize}
      \itemsep8pt
      \item Besides \LaTeX{} there are other formats
      \item While they are not as popular as \LaTeX{} they are in some ways more powerful
      \item Lua\LaTeX{} offers Lua support and can load every font you have installed on your system
      \item If you are interested, read this \link{https://www.overleaf.com/learn/latex/Articles/The_TeX_family_tree\%3A_LaTeX\%2C_pdfTeX\%2C_XeTeX\%2C_LuaTeX_and_ConTeXt}{article}.
   \end{itemize}
\end{frame}
