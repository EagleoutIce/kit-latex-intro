\section{Basic Formatting}

\subsection{Commands}

\begin{frame}[fragile]{What is a \say{Command}?}
\begin{itemize}
   \itemsep8pt
   \item \LaTeX-commands start with a backslash (\textbackslash)
{\color{lightgray}\begin{minted}[morekeywords={\\documentclass,\\begin,\\end},alsoletter={\\},morecomment={[l]{\%}}]{void}
\documentclass{article}
\begin{document}
Hallo Welt
\end{document}
\end{minted}
}
   \item Arguments are passed in curly braces (e.g., \T{article})
   \item (Line-)Comments start with a percent sign (\T{\%})
\end{itemize}
\end{frame}

\subsection{Shapes, Series, and Spacing}
\ltxpreview{BasicFormatting}
\begin{tcboutputlisting}
\documentclass{article}

\begin{document}
 Hello, \textbf{just \textit{great}} \texttt{format}.
\end{document}
\end{tcboutputlisting}

\begin{frame}[fragile]{First formatting commands}
   \begin{itemize}
      \itemsep8pt
      \item To write in \say{\textbf{b}old\textbf{f}ace}, \say{\textbf{it}alic}, or \say{\textbf{t}ele\textbf{t}ype}, use the corresponding commands:
      \eblLoadLtx{BasicFormatting}{graphics options={trim=4cm 24cm 11cm 4cm, clip}}{righthand width=.3\linewidth}
      \item As you can see, these commands can be nested (as long as the font supports this)
      \item Later, we will define our own commands for this  (e.g., \blatex[morekeywords={[5]{\\algorithm}}]{\\algorithm\{x\}} instead of \blatex{\\textbf\{\\textit\{x\}\}})
   \end{itemize}
\end{frame}

\ltxpreview{FirstDocument}% larger group to account for re-include
\begin{tcboutputlisting}
\documentclass{article}

\begin{document}
\textbf{1. Things I like} \\
- Penguins \\- Ducks \\

\textbf{2. Things I do not like}\\
Word, PowerPoint, and sentences deliberately stretched so they force a \textit{line break}.
\end{document}
\end{tcboutputlisting}

\begin{frame}{A simple document}
   \soldisablenumhl\begin{itemize}
      \item Together with \blatex{\\\\} which forces a new line we can create a simple document:
      \eblLoadLtx{FirstDocument}{graphics options={trim=4cm 21.5cm 3.25cm 4cm, clip}}{righthand width=6cm}
      \item However, please be aware that you should not force line breaks \say{real} document (there are better ways)
      \item Furthermore, you should refrain from using low-level markup-commands like \blatex{\\textbf} directly
   \end{itemize}
\end{frame}

\ltxpreview{SpacingExample}% larger group to account for re-include
\begin{tcboutputlisting}
\documentclass{article}

\begin{document}
Hello    World

Amazing! Sentence
\end{document}
\end{tcboutputlisting}
\begin{frame}{The importance of new lines}
   \begin{itemize}
      \item A simple line break will be transformed to a space
      \item Multiple spaces collapse to a single space
      \item An empty line starts a new paragraph (\blatex{\\par}) which is indented by default (by \blatex{\\parindent})
      \item Multiple empty lines will be collapsed to a single empty line
      \eblLoadLtx[columns=flexible]{SpacingExample}{graphics options={trim=4cm 23.5cm 10cm 4cm, clip}}{righthand width=6cm}
   \end{itemize}
\end{frame}

\subsection{Sections and Lists}

\ltxpreview{FirstDocumentWithToc}
\begin{tcboutputlisting}
\documentclass{article}

\begin{document}
\tableofcontents
\section{Things I like}
- Penguins \\- Ducks

\section{Things I do not like}
\subsection{Word!}
Word, PowerPoint,...
\end{document}
\end{tcboutputlisting}

\begin{frame}{Sections}
   \begin{itemize}
      \item Labeling your segments with \soldisablenumhl\blatex{\\textbf\{1. My Name\}} is very inconvenient
      \item For this, \LaTeX provides \blatex{\\section}, \blatex{\\subsection}, and more (depending on the documentclass)
      \item This allows you to create a table of contents as well!
      \eblLoadLtx{FirstDocumentWithToc}{graphics options={trim=4cm 17cm 3.5cm 4cm, clip}}{righthand width=6cm}
   \end{itemize}
\end{frame}

\begin{frame}{How \LaTeX{} produces the table of contents}
   \begin{itemize}
      \itemsep8pt
      \item When reading your document, \LaTeX{} collects all \blatex{\\section}, \blatex{\\subsection},~\ldots{} commands and writes them to a separate file
      \item If \LaTeX{} is run again it can use the information in this file (ending in \T{.toc}) to typeset the table of contents
      \item If you compile your document manually, you require a second run to update the table of contents
      \item Tools like \bbash{latexmk} or an IDE automates this for you
   \end{itemize}
\end{frame}

\ltxpreview{FirstDocumentItemize}
\begin{tcboutputlisting}
\documentclass{article}

\begin{document}
\section{Things I like}
\begin{itemize}
   \item Penguins
   \item Ducks
\end{itemize}
\end{document}
\end{tcboutputlisting}

\begin{frame}{Items and enumerations}
   \begin{itemize}
      \item Similar to sections, \LaTeX{} provides nicer ways to create lists
      \item For unordered lists, it offers the \blatex{itemize}, for ordered lists the \blatex{enumerate} environment
      \item Let's look at an example:
      \eblLoadLtx{FirstDocumentItemize}{graphics options={trim=4cm 23cm 9cm 4cm, clip}}{righthand width=6cm}
   \end{itemize}
\end{frame}

\subsection{Colors and Packages}
\ltxpreview{ColorExample}
\begin{tcboutputlisting}
\documentclass{article}
\usepackage{xcolor}
\definecolor{myOwnBlue}{RGB}{0, 100, 224}

\begin{document}
\textcolor{orange}{Orange Text!}
\textcolor{myOwnBlue}{Blue Text!}
\end{document}
\end{tcboutputlisting}

\begin{frame}{Colors}
   \begin{itemize}
      \item In order to use colors in \LaTeX{}, we want to use our first package!
      \item For this, we will use \T{\link{https://www.ctan.org/pkg/xcolor}{xcolor}} and load it with \blatex{\\usepackage\{xcolor\}}
      \item We can use predefined colors (\blatex{\\textcolor\{<color>\}\{x\}}), or define our own!
      \eblLoadLtx{ColorExample}{graphics options={trim=4cm 22.5cm 11cm 4cm, clip}}{righthand width=6cm}
   \end{itemize}
\end{frame}

\ltxpreview{PreambleExample}
\begin{tcboutputlisting}
\documentclass{article}
\usepackage[T1]{fontenc}
\usepackage[utf8]{inputenc}
\usepackage[english]{babel}
\usepackage{microtype}

\begin{document}
Hello!
\end{document}

\end{tcboutputlisting}
\begin{frame}{A default Preamble}
   \begin{itemize}
      \item Besides \link{https://www.ctan.org/pkg/xcolor}{xcolor}, there are several packages useful in every \LaTeX-Document
      \begin{itemize}
         \item \T{\link{https://www.ctan.org/pkg/inputenc}{inputenc}} and \T{\link{https://www.ctan.org/pkg/fontenc}{fontenc}} to ensure consistent encodings
         \item \T{\link{https://www.ctan.org/pkg/babel}{babel}} for advanced language support (hyphenation,~\ldots)
         \item \T{\link{https://www.ctan.org/pkg/microtype}{microtype}} for improved micro typography
      \end{itemize}
      \item Some of them take (optional) arguments in square brackets
      \eblLoadLtx{PreambleExample}{graphics options={trim=4cm 24cm 11cm 4cm, clip}}{righthand width=6cm}
   \end{itemize}
\end{frame}

\begin{frame}{Comments regarding your Preamble}
   \begin{itemize}
      \itemsep10pt
      \item With time, you will build your own preamble
      \item While several packages have a lot of package-options and some of them have interactions, you learn them rather quickly and on a need-to-use basis
      \item For example, if you want to configure babel for a german text, use \T{ngerman} (\say{new-german})
   \end{itemize}
\end{frame}
