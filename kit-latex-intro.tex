\RequirePackage{etoolbox}
\makeatletter
\appto\input@path{{logos/}{data/}}
\documentclass[en]{sdqbeamer}


\errorcontextlines99999
\begin{document}

%table of contents
\begin{frame}{Topic Overview}
\begin{columns}[c,onlytextwidth]
\column{.49\linewidth}
\tableofcontents[sections={1-4}]
\column{.49\linewidth}
\tableofcontents[sections={5-8}]
\end{columns}
\end{frame}

% \subimport{sections/}{01_getting_started.tex}
% \subimport{sections/}{02_basic_formatting.tex}
% \subimport{sections/}{03_math.tex}
% % \subimport{sections/}{04_environments.tex}
% \subimport{sections/}{05_figures_and_tables.tex}
% \subimport{sections/}{06_listings.tex}
% \subimport{sections/}{07_links.tex}
\subimport{sections/}{08_more.tex}

\appendix
\beginbackup

% \section{References}
% \begin{frame}[allowframebreaks]{References}
% \printbibliography
% \end{frame}

% \subimport{sections/}{10_additional.tex}

\section{Source}
\savebox\TempBoxA{\raisebox{-3pt}{\resizebox!{2\baselineskip}{\tikz{\pingu[feet=sit,eyes wink,sunglasses,lollipop left,bow tie, heart]}}}}
\begin{frame}{Source}
   \begin{itemize}
      \itemsep8pt
      \item The complete source code for these slides can be found on GitHub at:
      \begin{center}
         \link{https://github.com/EagleoutIce/kit-latex-intro}{https://github.com/EagleoutIce/kit-latex-intro}
      \end{center}
      \item No penguins were harmed in the making of these slides.
      \begin{center}
         \link{https://github.com/EagleoutIce/tikzpingus}{https://github.com/EagleoutIce/tikzpingus}
      \end{center}
      \item QR-Codes have been created with \FancyqrPkg!
   \end{itemize}\bigskip

   \begin{center}
      \fancyqr[l color=kit-purple!60!black,r color=kit-purple!60!kit-blue!60!black,height=3.25cm,image={\usebox\TempBoxA},level=H]{https://github.com/EagleoutIce/kit-latex-intro}
   \end{center}
\end{frame}
\backupend

\end{document}
