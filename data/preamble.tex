\PassOptionsToPackage{babel}{csquotes}
\RequirePackage{etoolbox}
\makeatletter
\appto\input@path{{logos/}{data/}{lib/}{lib/tikzpingus/tex/}{lib/fancyqr/}{lib/fancy-underline/}{lib/sopra-collection/}{lib/sopra-collection/sopra-listings}}
\documentclass[en,handout]{sdqbeamer}

\usepackage{import}

% use specifically the package in the submodule
\usepackage{lib/tikzpingus/tex/tikzpingus}
\usepackage{lib/fancy-underline/fancy-underline}
\makeatletter
\let\ULdepth\p@
\def\fancyulbackground{white}% so that it adapts with the beamer
\colorlet{url@line@color}{kit-blue!26!white}
\fancyulcolor{url@line@color}
% we make the auto detection of the depth bigger
\def\update@ul{\setbox\z@\hbox{{(j}}\edef\ULthickness{\the\dimexpr.235\dp\z@}}
% define a link macro that can be used with href and fancyul
\DeclareRobustCommand*\link{\hyper@normalise\link@do}
\def\link@do#1#2{\leavevmode\hskip-2\p@\href{#1}{\update@ul\hskip2\p@\fancyul{#2}}}% thinspace for some nicer padding
\DeclareRobustCommand*\hlink{\hyper@normalise\hlink@do}
\def\hlink@do#1#2{\leavevmode\hskip-2\p@\hyperlink{#1}{\update@ul\hskip2\p@\fancyul{#2}}}%

\AtBeginDocument{%
\long\def\beamer@ref#1{\hlink@do{#1}{{\beamer@origref*{#1}}}}% the extra set of braces is for soul so that it can work with braces
}
\makeatother

\usepackage[most]{tcolorbox}
\usepackage{lib/util/beamer-latex-preview}
\eblprefix{helper-}
\usepackage{lib/fancyqr/fancyqr}
\usepackage[encoding,print,highlights,nodefaultfont,fakeminted]{lib/sopra-collection/sopra-listings/sopra-listings}
\solSetNumSep{4pt}
\lstset{breakindent=2pt}
\solsetmintedstyle{plain number}
\solLoadLanguage{latex,bash}
\lstcolorlet{keywordA}{kit-blue}
\lstcolorlet{keywordB}{kit-brown!10!kit-gray}
\lstcolorlet{keywordC}{kit-blue70}
\lstcolorlet{command}{kit-blue}
\lstcolorlet{numbers}{kit-purple70}

\titleimage{X} % fake by load so we can have pingus :3
% we force scale for fancy(er) proprtions
\def\TitleLetter#1{\scalebox{2}{\textbf{\fontsize{100pt}{100pt}\selectfont#1}}}
\def\BannerLetter#1#2{\raisebox{-1cm}{\textbf{\fontsize{100pt}{100pt}\selectfont#2}}}

\let\say\enquote
\tikzset{@O/.style={overlay,remember picture}}
\usetikzlibrary{arrows.meta,decorations.pathreplacing}

\tcbset{
  lstpresenterstyle/.style={blankest,listing only,breakable=true,left=2.5mm}
}

\makeatletter
% https://tex.stackexchange.com/a/545948
\def\GlowWithColor#1#2#3#4{\pgfsys@beginscope% = pdf literal q
\pgfsetroundjoin% = 1 J
\pgfsetroundcap% = 1 j
\pdfliteral{1 Tr}% no pgf alternative
\foreach\ind in{#3,...,1}{%
    \pgfmathsetmacro\per{(11-\ind)*5}%
    \color{#1!\per}%
    \pgfsetstrokeopacity{#4}%
    \pgfsetlinewidth{1.75*\ind/#2}%
    \rlap{\Text}%
}%
\pgfsys@endscope}
\def\SetG{%
\def\Text{G}%
\GlowWithColor{kit-green!86!black!75!kit-purple!50!white}{1}{10}{.9}%
\GlowWithColor{kit-green!60!white}{2}{4}{.3}%
\GlowWithColor{kit-green!30!white}{2}{4}{.6}%
\Text}

\newsavebox\PHolder
\newsavebox\SHolder
\newsavebox\TTHolder
\newsavebox\SecondE
\newsavebox\SittingPengu
\def\TitleImageLoad#1{%% target: 4cm
\pingudefaults{eyes wink,body type=legacy,:mix-all=87!pingu@white!87!pingu@black}
\savebox\PHolder{\tikz{\pingu[xshift=2cm,yshift=-5mm,small,wings grab,headband=kit-red100,name=@p,staff left,staff left raise=-5.5mm, staff right,staff right raise=0mm,left item angle=-15,right item angle=-15]}}
\savebox\SHolder{\tikz{\pingu[:back,body=pingu@main!90!gray,wings wave,body type=chubby,hair=kit-orange100]}}
\savebox\SecondE{\scalebox{.7}{\rotatebox{30}{\TitleLetter{e}}}}
\savebox\SittingPengu{\tikz{\pingu[feet=sit, eyes shiny,halo,heart,small,body type=chubby]}}
\savebox\TTHolder{\tikz{\pingu[wings wave, banner, banner={tt}, eyes angry, body type=tilt-right, right foot sit,banner height=1.33cm,banner font=\BannerLetter]}}
\resizebox*!{4cm}{\begin{tikzpicture}
   \pingu[left wing wave,construction helmet, tie=kit-purple100,name=T]
   \pgfonlayer{very-background}
      \draw[gray] ([xshift=-6cm]T-foot-right) -- ++(50cm,0);
   \endpgfonlayer
   \node[above right=-16mm] (@T) at (T-wing-left-tip) {\TitleLetter{T}};

   \node[right=-5mm] (@Y) at(@T.south) {\rotatebox{200}{\TitleLetter{y}}};
   \pgfonlayer{background}
   \draw[gray,line cap=round] ([yshift=-3mm,xshift=1mm]@Y.north) -- ++(0,5.5cm) coordinate (@Top);
   \endpgfonlayer
   \node[right,yshift=3.3cm,xshift=7mm] (Y) at(@Y.south) {\rotatebox{-46}{\usebox\PHolder}};
   \pgfonlayer{background}
   \node[above,yshift=5mm,xshift=3.25mm] (@p) at(Y.east) {\TitleLetter{p}};
   \endpgfonlayer
   \node[right=-2.1cm,yshift=5.75mm] (@e) at(@p.south east) {\rotatebox{30}{\TitleLetter{e}}};
   \node[right=3mm,yshift=13mm] (s) at(@e.north) {\rotatebox{90}{\TitleLetter{\texttt{S}}}};
   \node[below] (@s) at (s.north|-@Top) {\rotatebox{180}{\usebox\SHolder}};

   \node[right=1cm,yshift=7.75mm,draw=gray,rounded corners=2pt,ultra thick,inner sep=1pt,outer sep=0pt] (@e2) at(@p.south east-|@s.east) {\usebox\SecondE};
   \node[above=-2mm] at(@e2.north) {\usebox\SittingPengu};

   \node[above right,yshift=-1.5mm,xshift=1cm] (tt) at(@e2.east|-T-foot-right) {\usebox\TTHolder};
   \node[right,yshift=-1.5mm] at(tt.east) {\TitleLetter{\rotatebox{-12}{I}\kern-16pt\smash{\(\Pi\)}\smash{\SetG}}};
\end{tikzpicture}}%
}

\title[\LaTeXe-Introduction]{An introduction to \LaTeXe}
\subtitle{Typesetting Articles, Reports, and Presentations}
\author{Florian Sihler (Ulm University)}

\grouplogo{SP-icon}
\grouplogowidth{1.5cm}% a little bit smaller
\institute{Institute of Software Engineering and Programming Languages (Ulm University)}

% Bibliography

\usepackage[citestyle=authoryear,bibstyle=numeric,hyperref,backend=biber]{biblatex}
% \addbibresource{templates/example.bib}
\bibhang1em

\AtBeginDocument{
\KITtitleframe
}

\newsavebox\TempBoxA